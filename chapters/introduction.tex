\chapter{Introduction}
The Standard Model (SM) is a spectacularly successful model of nature. It
correctly predicted the charm and top quarks, the gluon, and the W, Z and Higgs
bosons before their discoveries, and its numerical predictions agree with
experimental results to a good (and in some cases remarkable) degree of
precision. Nevertheless, the SM does not describe everything we observe: there
is not enough charge parity violation to produce the imbalance between baryonic
and antibaryonic matter in the visible universe, there is no good candidate for
dark matter, and it does not address gravity.  Furthermore, some believe the SM
is not entirely satisfactory because it depends on many arbitrary parameters,
and it does not explain, for example, the origins of electroweak symmetry
breaking and the reasons underlying the existence of three generations of
particles and the masses of those particles. The SM includes all of the
currently known particles, but it is possible that more particles are yet to be
discovered. Such particles may not have been detected because the probability
that they would interact with currently known particles is small, or they may be
so massive that the energy needed to produce them exceeds the amount that
current experiments can provide. The SM may only be an effective low-energy
theory, an approximation of reality that is valid at scale $\Lambda$. Through
the use of effective field theory (EFT), potential deviations from the SM due to
currently unknown particles or forces (new physics, or NP) can be parameterized
in a model-independent way by extending the SM Lagrangian with higher-dimension
operators.

NP involved in electroweak symmetry breaking would have a large coupling to the top
quark because it is the heaviest particle. This dissertation describes two
measurements of top-quark pairs produced in association with W or Z bosons (\ttW
and \ttZ). These measurements are interpreted within the framework of EFT in
order to constrain the Wilson coefficients of selected dimension-six operators;
these coefficients characterize the strength of the NP interaction.

This dissertation is organized as follows: the theoretical framework is introduced
in~\cref{chap:theory}. The Large Hadron Collider (LHC) and the Compact Muon
Solenoid (CMS) experiment are described in~\cref{chap:apparatus}.
\Cref{chap:objects} explains how particle observables are reconstructed from the
raw detector electrical signals and the quality requirements that are imposed on
them. \Cref{chap:8-TeV,chap:13-TeV} present cross section measurements of \ttW
and \ttZ production performed using LHC data from proton--proton collisions
collected during Run 1 at $\sqrt{\text{s}}=\eightTeV$ and Run 2 at
$\sqrt{\text{s}}=\thirteenTeV$, respectively. \Cref{chap:eft} is the main
subject of this dissertation. It describes the interpretation of the \eightTeV and
\thirteenTeV \ttW and \ttZ measurements within the framework of EFT in order to
constrain the Wilson coefficients of selected dimension-six operators that might
signal the presence of NP contributions in \ttW and \ttZ production.

\Cref{chap:theory,chap:apparatus,sec:particle-flow} are mainly reviews of
existing literature. I made significant contributions to the analysis described
in~\cref{chap:8-TeV}, but not to the analysis described in~\cref{chap:13-TeV},
and was the main analyst for the EFT interpretation described
in~\cref{chap:eft}. The \eightTeV measurements and their EFT interpretation have
been published in the Journal of High Energy Physics~\cite{JHEP-1601-2016-096}.
The \thirteenTeV measurements and their EFT interpretation have been submitted
to the Journal of High Energy Physics~\cite{TOP-17-005}.
