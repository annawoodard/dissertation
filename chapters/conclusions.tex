\chapter{Conclusions}
In the face of limited collider energy, and as researchers looking for striking
evidence of NP in traditional ``bump hunts" at the LHC continue to come up
empty-handed, it is necessary to find new ways of searching for NP. To that end,
the main goal of this dissertation was to interpret measurements of \ttW and
\ttZ within the model-independent framework of effective field theory to
constrain the Wilson coefficients of dimension-six operators that would affect
\ttW and \ttZ. Effective field theory interpretations are challenging because
the large phase space of possible Wilson coefficient values make traditional
modes of analysis untenable. In this dissertation, this difficulty was managed
by making the simplifying assumption that NP has a negligible effect on the
kinematics of studied processes and only scales their overall production. We
determined a parameterization of the cross section scaling due to NP effects
which could be evaluated as a substitute for producing dedicated samples at each
studied point in the phase space of possible Wilson coefficient values. The
first iteration of this analysis used LHC Run 1 proton-proton data at
$\sqrt{\text{s}}=\SI{8}{\TeV}$ to set constraints on the Wilson coefficients of
five dimension-six operators. Building on lessons learned during the \eightTeV
analysis, we used LHC Run 2 proton-proton data at
$\sqrt{\text{s}}=\SI{13}{\TeV}$ to set constraints on eight dimension-six
operators. The second iteration of the analysis incorporated several
improvements. We used a more sophisticated method of selecting which operators
to study, made improvements that increased the quality of the parameterization,
and removed couplings to the first two generations from the NP model. Both the
\eightTeV and the main part of the \thirteenTeV analyses studied the special
case in which only one Wilson coefficient is enhanced at a time, while the
others are fixed at zero. As a first step towards global constraints,
preliminary results of simultaneous fits for pairs of coefficients and to the
Wilson coefficients of all eight selected operators were also presented.

For a given process, many combinations of Wilson coefficient values may
correspond to identical scaling of the cross section. These surfaces of equal
scaling are a fundamental degeneracy that cannot be resolved using the approach
taken in this dissertation. Nevertheless, our work has hinted at promising
directions for future investigation. We carefully studied the effects of various
operators on \ttW and \ttZ background processes and we identified several
operators that had a large effect on expected background yields. We found
particularly large effects for triboson processes, ZH, WH, and the associated
production of a single top quark and a Z or Higgs boson. Future work should add
dedicated signal regions for these processes. Because the NP effects vary from
process to process, additional sensitivity would be gained from performing a
simultaneous fit with these additional signal regions.

Building on our work developing reliable parameterizations of the scaling
effects of NP on cross sections, we have presented a technique for
parameterizing event weights that can be used to model the effects of various
combinations of Wilson coefficient values with a single simulation sample. We
have demonstrated good agreement between the weights derived from the
parameterization and the weights calculated in the traditional manner. Future
work could use this strategy to perform a wide variety of detector-level
analyses, taking advantage of sophisticated techniques for exploiting kinematic
discriminants which would further aid in resolving degeneracies. As data
accumulate and analysis techniques are refined, stronger limits on NP coupling
parameters or the discovery of NP effects will be possible.
