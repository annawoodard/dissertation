
\RequirePackage{luatex85}
\documentclass[tikz, border=10pt]{standalone}

\usepackage[compat=1.1.0]{tikz-feynman}

\newcommand{\Ptop}{\ensuremath{\text{t}}}

\makeatletter
\tikzset{
  position/.style args={#1 degrees from #2}{
    at=(#2.#1), anchor=#1+180, shift=(#1:\tikz@node@distance)
  }
}
\makeatother

\begin{document}

\tikzfeynmanset{
  proton/.style={
    draw=black,
    preaction={
      draw=black,
      double,
      double distance=3pt
    },
    /tikzfeynman/with arrow=0.5,
    % /tikz/postaction={
    %   /tikz/draw,
    %   /tikz/
    %   /tikzfeynman/with arrow=0.5
    % }
  }
}
\begin{tikzpicture}
  \begin{feynman}
    \diagram [horizontal'=hi to hf,tree layout,sibling distance=12mm,level distance=20mm] {
      % Higgs
      hi -- [draw=none] t1f,
      hi -- [scalar,edge label'=\(H\)] hf,
      hi -- [draw=none] t2f,
      % τ 1
      hf -- [fermion,edge label'=\(τ^-\)] t1
         -- [fermion] n1 [particle=\(\nu_\tau\)],
      t1 -- [boson,edge label=\(W^-\)] w1
         -- [fermion] f1 [particle=\(q\)],
      w1 -- [anti fermion] f2 [particle=\(\overline q'\)],
      % τ 2
      hf -- [anti fermion,edge label=\(τ^+\)] t2
         -- [boson,edge label'=\(W^+\)] w2
         -- [fermion] f3 [particle=\(q\)],
      w2 -- [anti fermion] f4 [particle=\(\overline q'\)],
      t2 -- [anti fermion] n2 [particle=\(\overline \nu_\tau\)],
      % top
      t1f -- [anti fermion] b1f [particle=\(\overline b\)],
      t1f -- [boson,edge label=\(W^-\)] w1f -- [anti fermion] j1 [particle=\(\overline q'\)],
      w1f -- [fermion] j2 [particle=\(q\)],
      % other top
      t2f -- [boson,edge label'=\(W^+\)] w2f -- [fermion] l2 [particle=\(\Pnu_\ell\)],
      t2f -- [fermion] b2f [particle=\(b\)],
      w2f -- [anti fermion] l1 [particle=\(\ell^+\)],
    };

    \vertex[above left=30mm and 6mm of hi] (t2i);
    \vertex[blob,above left=25mm and 20mm of t2i] (p2) {};
    \vertex[particle=\(p\),left=25mm of p2] (p2i) {\(p\)};
    \vertex[above right=5mm and 25mm of p2] (p2f);

    \vertex[below left=30mm and 6mm of hi] (t1i);
    \vertex[blob,below left=25mm and 20mm of t1i] (p1) {};
    \vertex[particle=\(p\),left=25mm of p1] (p1i) {\(p\)};
    \vertex[below right=5mm and 25mm of p1] (p1f);

    \diagram* {
      {[edges=proton]
        (p1i) [particle=\(p\)] -- (p1) [blob] -- (p1f),
        (p2i) [particle=\(p\)] -- (p2) [blob] -- (p2f),
      },
      (p1) [blob] -- [gluon,edge label=\(g\)] (t1i),
      (p2) [blob] -- [gluon,edge label'=\(g\)] (t2i),
      (hi) -- [anti fermion] (t1i) -- [anti fermion,edge label=\(\overline t\)] (t1f);
      (hi) -- [fermion] (t2i) -- [fermion,edge label'=\(t\)] (t2f);
    };
  \end{feynman}
\end{tikzpicture}
% \subfloat[][]{
  % \feynmandiagram [horizontal=l2 to h] {
  %   g1 -- [gluon] l1 -- [anti fermion,edge label={t,b}] l2 -- [scalar, edge label=H] h,
  %   g2 -- [gluon] l3 -- [fermion] l2,
  %   l3 -- [anti fermion] l1,
  %   g1 -- [draw=none] g2,
  % };
% }
  % \feynmandiagram [horizontal=v1 to v2] {
  %   f1 [particle=\(f\)] -- [fermion] v1 -- [boson] v2 -- [boson] v3 [particle={\PW,\PZ}],
  %   f2 [particle=\(\overline f\)] -- [anti fermion] v1,
  %   v2 -- [scalar, edge label'=\PH] h1,
  % };
  % \label{sfig:hs}
% }

  % \feynmandiagram [horizontal=h1 to h2] {
  %   f1 -- [fermion] i1 -- [fermion] f1f [particle=\(f\)],
  %   i1 -- [boson] h1 -- [boson] i2,
  %   h1 -- [scalar, edge label'=\PH] h2,
  %   f2 -- [fermion] i2 -- [fermion] f2f [particle=\(f\)],
  %   f1 -- [draw=none] f2,
  %   i1 -- [draw=none] i2,
  %   f1f -- [draw=none] h2,
  %   f2f -- [draw=none] h2,
  % };
  % \label{sfig:vbf}
% }
  % \feynmandiagram [horizontal=h1 to h2] {
  %   f1 -- [gluon] i1 -- [anti fermion] f1f [particle=\(\overline \Ptop\)],
  %   i1 -- [fermion] h1 -- [fermion] i2,
  %   h1 -- [scalar, edge label'=\PH] h2,
  %   f2 -- [gluon] i2 -- [fermion] f2f [particle=\(\Ptop\)],
  %   f1 -- [draw=none] f2,
  %   i1 -- [draw=none] i2,
  %   f1f -- [draw=none] h2,
  %   f2f -- [draw=none] h2,
  % };
  % \label{sfig:ttH}
% }
% \begin{tikzpicture}
%   \begin{feynman}
%     \diagram [arrow size=1pt, vertical=a to b] {
%       i1 [particle=\(\mbox{g}\)] -- [gluon] a -- [fermion] c [blob, minimum size=0.4cm],
%       a -- [anti fermion, edge label'=\(\mbox{t}\)] b,
%       f1 [particle=\(\bar{\mbox{t}}\)]-- [fermion] b -- [gluon] f2 [particle=\(\mbox{g}\)],
%     };

%     \vertex [position=45 degrees from c] (d) {\(\mbox{t}\)};
%     \vertex [position=-45 degrees from c] (e) {\(\mbox{Z}\)};

%     \diagram* [arrow size=1pt] {
%       (a) -- [fermion, edge label=\(\mbox{t}\)] (c),
%       (d) -- [anti fermion] (c) -- [boson] (e);
%     };
%   \end{feynman}
% \end{tikzpicture}
\end{document}
