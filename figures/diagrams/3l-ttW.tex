\tikzfeynmanset{
  small,
}
\begin{tikzpicture}
  \begin{feynman}
    \diagram [horizontal'=v1 to v4,tree layout,sibling distance=6mm,level distance=15mm] {
      v1 -- [anti fermion,edge label'=\(\overline \Ptop\)] v3
         -- [anti fermion] p1 [particle=\(\overline \Pb\)],
      v3 -- [boson,edge label=\(\PW^-\)] v5
         -- [fermion] p2 [particle=\(\Plep^-\)],
      v5 -- [anti fermion] p3 [particle=\(\overline \nu\)],
      v1 -- [fermion,edge label=\(\Ptop\)] v2
         -- [boson,edge label'=\(\PW^+\)] v4
         -- [fermion] p4 [particle=\(\nu\)],
         v4 -- [anti fermion] p5 [particle={\(\Plep^+\)}],
      v2 -- [fermion] p6 [particle=\(\Pb\)],
    };

    \vertex[left=20mm of v1] (v6);
    \vertex[above=20mm of v6] (v7);
    \vertex[above left=4mm and 20mm of v7] (p7) {\(\overline \Pq^\prime\)};
    \vertex[above right=4mm and 20mm of v7] (v8);
    \vertex[above right=4mm and 20mm of v8] (p8) {\(\nu\)};
    \vertex[below right=4mm and 20mm of v8] (p9) {\(\Plep^+\)};
    \vertex[below left=4mm and 20mm of v6] (p10) {\(\Pq\)};

    \diagram* {
      (p7) -- [anti fermion] (v7) -- [anti fermion] (v6),
      (p10) -- [fermion] (v6) -- [gluon] (v1),
      (p8) -- [anti fermion] (v8) -- [anti fermion] (p9),
      (v8) -- [boson, edge label=\(\PW^+\)] (v7),
    };
  \end{feynman}
  \begin{pgfonlayer}{background}
      \node[fill=blue,fill opacity=0.4,circle,inner sep=1mm,fit=(p9)] {};
      \node[fill=blue,fill opacity=0.4,circle,inner sep=1mm,fit=(p5)] {};
      \node[fill=blue,fill opacity=0.4,circle,inner sep=1mm,fit=(p2)] {};
  \end{pgfonlayer}
\end{tikzpicture}
